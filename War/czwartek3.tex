\documentclass{article}

\usepackage{a4wide}
\usepackage{polski}
\usepackage[utf8]{inputenc}
\usepackage{float}
\usepackage{warhammer}
\usepackage[ampersand]{easylist}
\usepackage{indentfirst}

\title{\Huge{Chaos i chaos}}
\author{Arkadiusz Popczak}
\begin{document}
\maketitle

\section*{Wydarzenia w mieście i następstwa działań graczy/gildii i straży}
    Gracze odkąd pojawili się w mieście byli uczestnikami i świadkami wielu zmian i działań. Miasto wygląda inaczej niż na początku oraz poprzez działania wielu podmiotów każda organizacja jest teraz w innym stanie niż na początku.

\subsection*{Miasto}
    Aktualnie w mieście zbliża się wielki festiwal. Euforia i dobry nastrój jaki panowały na tych festiwalach stłumiony jest przykrą atmosferą. Ludzie są coraz mniej chętni do rozmawiania a rzeczy które dochodziły do słuchu mieszkańców napawały ich niepewnością i nieufnością. Każdy test plotkowania/przekonywania na mieszkańcach jest utrudniony o 10. Następstwa spalenia banku, portu Sail $\&$ Sound, częściu willi Brightkeepera i jednego budynku w górnym mieście spowodowały wprowadzenie dodatkowych podatków na "reperację szkód". Dodatkowo zaczyna powstawać niesamowita inflacja, z powodu zatrzymania dostaw materiałów i żywności do i z miasta. Produkty co tydzień będą drożały o kolejne 100$\%$ w stosunku do ceny z podręcznika  (Ten tydzień zaczyna się na 200$\%$). W slumsach zaczyna brakować straży a brak Barona pozwolił rozprzestrzenić się tam niezorganizowanej przestępczości. Ludzie idący tam zaczynają znikać. Każdy działa na własną rękę. Znikają też strażnicy miejscy znajdowani potem w rynsztokach czy nie znajdywani.

    Szepczący postanowili wykorzystać sytuację aby z pomocą Barona wrócić na swoje włości i dalej prowadzić eksperymenty na ludziach. W środkowej części miasta gildia złodzji wraz z Srebrnym Syndykatem zaczęła okradać co bogatszych mieszkańców i składować dobra na kryzys który ma nadejść.

    W części miasta przeznaczonej dla podróżnych szkoły walki rekrutują chętnych do walki na arenie. Za drobną opłatą możesz dać sobie szansę na chwałe i zwycięstwo lub śmierć w bitwie. Sami także będą startować w walkach. Coraz więcej jest obcych osób i tłumów na ulicach. Testy kieszonkowe są polepszone o 10 w tej sekcji miasta. Wiąże się to także, że coraz więcej osób jest uzbrojonych więc po zmroku można trafić na nieprzyjemnych panów pytających czy mogą pomóc wydać Ci pieniądze.

\subsection*{Gildia}
    Od czasu przybycia graczy w komórka gildii w tym mieście bardzo osłabła. Zaufani którzy dla gildii pracowali albo umarli, zdradzili ewentualnie oszaleli. Fundusz operacyjny który każdy dostawał nie uszczupł jeszcze bradzo z powodu prywantych skrytek z kasą. Osoby które pozostały:\\

    \begin{easylist}
        & Kerl	Xurgug
        & Rose	Johnson
        & Gangolf	Berlepsch
        & Ida	Feig
        & Drirulugg 	Metalgrog
        & Liwanferi	The Fearless\\
    \end{easylist}

    Z takimi zasobami, brakiem porządnych kandydatów oraz problemami z operacyjnością niektórych, Kerl chciał wybrać się na zebranie rady przewodniczących lokalnych komórek aby poprosić o wsparcie i przedstawić dokładnie sytuacje w mieście. Niestety został zatrzymany przez ważne sprawy, o których poinformowali go BG. Do gildii niedługo dojdzie jeden zaufany który ma zająć się misjami poza miastem w zastęstwie za zdradzieckiego Steelragea. Jeżeli prośba Kerla zostanie spełniona wróci on z dodatkowymi środkami (ludźmi i pieniędzmi). Potrwa to dwa tygodnie. W tym czasie nad zaufanymi przejmie kontrolę Liwanferi. Z notatek Irmgard przekazanych do Kerla, wcześniej wspomniany wyciągnął informacje o Karlu Mittagu i jego powiązaniach z Tillichem i Irmgard. Założyli, że bardziej pomocne będzie zagrożenie mu niż wystawienie do straży. Trzymają to w sekrecie przed strażą.

    W Talabeklandzie jest jeszcze 4 komórki i siedziba W Talabaheim. Tam udawał się Kerl na spotkanie z przełożonym.

\subsection*{Straż}
    Po wydarzeniach w slumach do straży przyjęto nowych niewyszkolonych strażników w celu uzupełnienia szeregów. Brak dyscypliny oraz wyszkolenia powoduje wiele nieprzyjemnych sytuacji między strażą i mieszkańcami. Powoduje to także nieprzyjemności w samej straży. Zbliżający się festiwal pochłania znaczne siły straży w utrzymaniu porządku w mieście. Dodatkową czynnikiem entuzjazmu w War District jest zbliżająca się walka o etat na oficera straży, który miał zastąpić Hansa Krugera. Straż ma problemy z opanowaniem sytuacji w slumsach oraz w środkowym mieście i dzielnicy pordóżniczej.
    Nie pomaga także dla wizerunku straży spowodowanie pożaru na skutek którego spłoneły budynki portowe spółki Sail $\&$ Sound. Podczas pobytu BG w dzielnicy Wojskowej właściciel spółki zabezpieczył pancerne cele blokując zamki na stałe oprócz celi ze Smokoogrem. Dodatkowo wzmocnione patrole w każdej dzielnicy mocno uszczuplają możliwości natychmiastowej reakcji straży w kryzysowych sytuacjach. W tym momencie dostawy dla straży także zostaną wstrzymane i podrożeją.

\subsection*{Baron}
    Po ucieczne z miasta wraz z synem i kilkuosobową ekipą Baron udał się do okolicznych miast aby zebrać z tych miast swoich agentów. Po zebraniu odpowiedniej ilości ludzi udał się do posiadłości Irmgard gdzie zastał zgliszcza i płomienie. Domyślając się kto mógł to zrobić podją działania zapobiegawcze. W tym czasie gdy Baron odzyskiwał siłi i kontaktował się z szepczącymi po wsparcie. Zbierając kosztowności oraz pieniądze od Irmgard powoli odbudował swoich bandziorów. Mniej więcej w tym momencie poprzez dwóch członków zaufanych gildii (szpiegów Barona) udało spalić mu się bank wraz z zasobami gildii (Brak weksli, przyjmowanie zapłat tylko w gotówce i marna sytuacja finansowa komórki). Celem barona jest teraz wywołanie zamieszek w mieście aby osłabić straż i przejąć władzę nad co najmniej slumsami. Po skontaktowaniu się z BG baron przekazał im pewne informacje o gildii oraz szlachcie i radzie. Gracze zdecydowali się nie współpracować z Baronem, jednocześnie deklarując się wrogami. 

\subsection*{Rada}
    Rada już wie kto bruździ im w ich spokojnym i dostatnim życiu gdzie nikt wcześniej im nie przeszkadzał (oprócz Barona i Gildii). Rada jest osłabiona i zaczyna poważnie podchodzić do tematu ataków na szlachtę i członków rady. W ostatnich dwóch tygodniach zginęła Irmgard oraz został zatrzymany Brightkeeper. Na ich szczęście Mittag szybko przygotował dokumenty uniewinniające choć na chwilę szlachcica. Od tamtego czasu Rada zarzuciła przygotowania do festiwalu (czytaj co się dzieje w mieście) i zaczęła wynajmować skrytobójców i szpiegów do inwigilacji i ewentualnego pozbycia się problemów. Po mieśnie została rozesłana notka z na chwilę obecną szczątkowymi informacjami na temat grupy BG. W dodatku Rada zaczęła zatrudniać prywatną ochronę rezygnując powoli z ochrony straży, która okazała się nieskuteczna (rabunki, podpalenia, zabójstwa). 


\section*{Historia trzech przyjaciół}
    Wszystko zaczeło się jakieś 30 lat temu w górach szarych kiedy to Kerl wyruszył z rodzinnej twierdzy Karak Azgal w poszukiwaniu chwały i pieniędzy aby móc wrócić do twierdzy jako wojownik! I tak się złożyło, że przy pierwszym starciu z leśnym trollem dostał nie mały łomot. Uratowała go tam drużyna która na tego trola dostała zlecenie w okolicznej wiosce. W drużynie był Baron oraz Bel-Venzel. Po szybkim leczeniu i opatrywaniu ran razem udali się na trola. Tym razem z sukcesem. Baron i Bel zauważyli, że potrzebują pomocy czasem na misjach i jadą w góry czarne na polowanie na chaos. Krasnolud właśnie czegoś takiego szukał. W podróży poznał, że baron pochodzi ze szlacheckiej rodziny i podróżuje po świecie w poszukiwaniu czegoś co przyniesie rodowi większą chwałą. Bel został wygnany kilka set lat temu ze swojego lasu i od tamtej pory tuła się po świecie. Spotkali się z baronem podczas rozwiązywania zagadki zaginionej słotej zastawy pewnej szlachcianki. I tak razem polowali i zabijali bestie przez kilka lat. Baron zaobserwował że chaos walczy sam ze sobą. Bel chciał się w końcu zatrzymać a Kerl wrócić w chwale do swoich ziem. Zostali zauważeni przez Nameless w Talabaheim. Tam przedstawiono im metody zarobku i zmiany systemu. Pomimo braku początkowego zainteresowania Kerl z Baronem dołączyli do gildii zaś Bel osiadł się w mieście Krugenheim.

    Potem już wszystko jasne. Baron miał wkupić się w łaski szlachty ale nie podobało mu się to, że gildia też uderza w niego. Poczuł smak władzy i skończył tam gdzie skończył. Bel został generałem ze swoim doświadczeniem broniąc miasto i okolice przed chaosem a Kerl awansował w gildii oraz stał się głową komórki.

\section*{Chaos w mieście}

    TBD - punkty które mają się wydarzyć, ale nie opisuję fabuły.

\section*{Potwory na wolności}

    TBD - punkty które mają się wydarzyć, ale nie opisuję fabuły.

\section*{Powrót Barona}

    TBD - punkty które mają się wydarzyć, ale nie opisuję fabuły.

\section*{Nowa władza}

    TBD - punkty które mają się wydarzyć, ale nie opisuję fabuły.

\end{document}