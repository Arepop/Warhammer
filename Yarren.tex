\documentclass{article}

    \usepackage{a4wide}
    \usepackage{polski}
    \usepackage[utf8]{inputenc}
    \usepackage{float}
    \usepackage{xcolor}
    \usepackage[ampersand]{easylist}
    \usepackage{tcolorbox}
    \usepackage{graphicx}
    \usepackage{transparent}
    \usepackage{easylist}

    % \usepackage[greek]{babel}
    % \usepackage[LGR]{fontenc}

    \newcommand{\paragraphx}[1]{
        \paragraph{\Large{{#1}}}\mbox{}

    }

    \newcommand{\textbb}[1]{
        \smallskip
        \textbf{{#1}}
        \smallskip
    }

    \title{Yarret Birgim III - historia}
    \author{Arkadiusz Popczak}
    \date{}
    \begin{document}
    \maketitle

    Historia naszego bohatera zaczyna się spokojnie. Jako 5 dziecko w swojej rodzinie należało do starszego rodzeństwa. Jak każdy wie, niziołkowie i ich wielodzietne rodziny to całkowicie normalna rzecz. Wraz z 7 młodszego rodzeństwa w domu rodzinnym było ich 13. 3 starszych braci, starsza seiostra, 3 młodszych braci i 4 młodsze siostry. Rodzice nadali mu imię po wuju, Yarret III. Najlepszymi wspomnieniami z dziećiństwa były zabawy w chowanego, podkradanie jabłek od sąsiada oraz nauka czytania i pisania od starszej siostry. Ten stan rzeczy utzymywał się do 9 urodzin Yarreta kiedy jako niziołek poznał w sobie talent magiczny. Stało się to kiedy kradł jabłka od sąsiada wraz z braćmi i kiedy zauważył, że jałko które ukradł zaminiło się w gruszkę.\\

    Rodzice Yarreta wysłali go do wuja immiennika, który od lat pracował dla państwa jako tkacz. Prowadzący zakład tkacki wuj (Yarret II, który nosił je po pierwszym Birgimie i założycielu rodu) często w swoim zkładzie był odwiedzany przez niesforne dzieciaki, na które krzyczał kiedy dostawał kolejne zlecenie na szaty dla magów czy akademików, lecz w głębi serca cieszył się ich radością. Po przejrzeniu kilku ksiąg oraz sprawdzeniu kilku ciekawych faktów dotyczących czarodziejów niziołków, stwierdził, że takiej okazji jak ta nie można przegapić. Tego samego dnia udał się do dyrektora szkoły gdzie po długiej rozmowie trwającej 5 tys. złotych monet Yarret został przyjęty na pierwszy rok.\\

    Przez kolejne lata Yarret z dziecka które myślało tylko dniem dzisiejszym musiał skupić się na nauce języków (Thayan, Mulhorandi, Tuigan, Chessentan etc.), rozwijaniu talentu magicznego, poznawaniu historii kraju i jego struktury. Jego sytuacja nie należała do najlepszych, gdyż ewenement jakim jest niziołem mag nie sprzyjał w zawieraniu znajomości. Te rozwineły się dopiero kiedy pod koniec edukacji początkowej nasz bohater w bibliotece znalazł starą i zakurząną księgę z wypisanymi zaklęciami. Nie wszystkie z nich rozumiał, ale te które rozumiał szczególnie przypadły mu do gustu. Może nazwiecie to oszustwem, lecz dzieki niej niziołek zakończył podstawową edukację po tradycyjnych 6 latach jako jeden z lepszych w swej szkole.\\

    Po wspaniałym rozdaniu dyplomów Yarret złożył podanie o przyjęcie do najlepszej akademii w Thay, Akademii Bezanthur. Konkurencja dla niziołka okazała się za duża. Jego podanie zostało odrzucone a on sam nie wiedząc co ma zrobić popadł w depresję. Jego marzenia o zostaniu czerwonym magiem niziołkiem w tym momencie staneły w miejscu. Z tego stanu wyciągnęli go bracia proponując mu aby zaciągną się na termin u wuja a z odłożonych tam pieniędzy zapisał się na prywatne lekcje do Wielkiego Maga Pyrados. Kiedy to czytacie możecie nie uwierzyć, że to mu się udało. I macie całkowitą rację. Yarretowi się to nie udało. Ale dzięki pracy w zakładzie mógł zarobić pieniądze na wyjazd do Bezanthur. Dwa lata na terminie u wuja nauczyły go tkactwa i po podejściu do egzaminu czeladniczego zdał go bez najmniejszych problemów. Z takimi kwalifikacjami i odłożonymi pieniędzmi wyjechał do Bezanthur aby tam znaleźć nauczyciela magii, który pomoże mu dostać się do akademii. Podczas pożegnania z rodziną ojciec niziołka rzekł do niego:\\

    - Yarrecie Birgimie III,  jako twój ojciec błogosławię Ci w twoich celach i życzę Ci jak najlepiej. Jesteś wyjątkowym niziołkiem dla naszego rodu i swe imię noś z szacunkiem i dumą gdyż otrzymawszy je po wuju pochodzi ono od naszego pra-pra dziada który nasz ród założył, sprowadzając się w te rejony po wojnie. Weź proszę jeszcze ten pierścień, który do niego należał. (Pierścień złoty z wypisanym po wewnętrznej stronie mottem: Morituri Nolumus Mori. tł: My którzy mamy za chwilę umrzeć, nie chcemy.)\\

    Po takim pożegnaniu Yarret ze łzami w oczach wyruszył do największego miasta Thay. Wtedy jeszcze nie wiedział, że jego ojciec będzie żył długo i umrze we własnym łożu z o jejku... Bardzo liczną rodziną... (Nie zawsze historia musi być tragiczna =D). Po dotarciu do stolicy i znalezieniu pokoju na strychu u pani Maggabet, która prowadziła sklep zielarski w którym ochoczo podczas studiowania ksiąg Yarret pomagał. W sklepie tym zachodziło sporo osób a samotne poszukiwania mistrza magii w mieście nie szły za dobrze. W międzyczaise pomiędzy nauką i pomocą w sklepie Yarret przyjmował zlecenia jako tkacz a po czasie znalazł także mistrza tkactwa Rincewinda (człowiek) nieopodal sklepu zielarskiego, gdzie podją się majstersztyku (tak nazywała się praca czeladnika który miał podejść do egzaminu na mistrza tkactwa).\\

    Rincewind był tkaczam, ale co ważniejsze, był wykładowcą na Akademii Bezanthur gdzie uczył wojennej sztuki wsparcia. Nie był to mistrz ani miły, ani pomocny. Tak surowego nauczyciela jeszcze niziołek nie miał. Przymuszany do nauki języków tajemnych, bo klienci z różnych krain przychodzą i trzaba się dogadać. Zaawansowane techniki tkactwa, które kazał opanowywać do percekcji, mimo ran na dłoniach Yarreta. Surowość okazywał także w nauce etykiety, bo jak mozna mówić "Cześć" czy "Ej Ty". Mag nie przyznał się swojemu czeladnikowi, że za
    kład ten jest jedynie aby szlifować swoje umiejętności tkania "Nadprzyrodzonych Gobelinów" (Eldritch Tapestry). Talent u niziołka wyczuł od razu. Nauka języków nie służyła aby gadać z klientami, ale Rincewind dawno nie miał żadnego ucznia na akademii który tak bardzo starał by się w dążeniu do swojego celu. Po dwóch latach nauki do egzaminów wstępnych i pracy w sklepie zielarskim, Rincewind zadowolony z postępów chłopaka podał go na egzamin mistrza tkactwa. Komisja która egzaminowała Yarreta była pod wrażeniem wyćwiczenia i zdolności jakie prezentował egzaminowany. I z tytułem mistrza tkactwa złożył podanie na akademię. Ku jego zdzinwieniu już kolejnego dnia dostał odpowiedź aby wieczoram stawił się na egzamin wstępny do Akademii Bezanthur. Ubrawszy swój najlepszy strój wyjściowy, z pelną werwą i nutą stresu Yarret udał się na egzamin. W przedsionku do sali egzaminacyjnej oprócz niego czekało jeszcze dwóch kandydatów.\\

    Kandytat 1: Słyszałem, że dzisiaj odrzucili każdego na egzaminie. Podobno są niesamowicie wymagający i nie przyjmują byle kogo.

    Kandytat 2: Rok temu też starałem się dostać. Odrzucili mnie jedynnie dlatego, że nie podobała im się moja postawa podczas rzucania zaklęć. Zbyt mało dostojna.

    Kandytat 1: O popatrz... Niziołek. Tego to na pewno nie przyjmą. <smiech>

    Kandydat 2: Hej! Kurdupel! Chyba pomyliłeś sale. Pszedszkole jest po drugiej stronie miasta <smiech>

    Yarret: Właśnie z tamtąd przychodzę panowie. Szukają dwóch pajaców do rozbawiania dzieci. Myślę, że się nadajecie.

    Kandydat 1: Uuu, gryzie. Jestem Vincent. A kolega to Coen.

    Yarret: Miło panów poznać.\\

    W tym momencie z drzwi od sali otworzyły się nagle a ze środka wybiegła zapłakana dziewczyna, odtrącając trzech nowo poznanych kumpli. Z sali dobiegł głos znajomy dla Yarreta, ale przez stres nie do końca umiałgo zidentyfikować. Głos ten wołał Vincent Gurney proszony na egzamin. Vincent wszedł na salę. Po 20 minutach. Drzwi otworzyły się a ze środka dobiegło wołanie kobiecego głosu. Coen Barba. Tak samo jak z Vincentem, Coen wszedł na salę. Po kolejnych 20 minutach sytuacja się powtórzyła tym razem wojając na salę Yarreta Birgima III. Po wejściu na wielką salę na której znajdowało siędługie potężne drewniane biurko, niziołek zauważył komisję. W komisji składającej się z trzech ludzi: Vienna McRay, wykładowczyni "Run i symboli magicznych", Harry Niew wykładowca "Magi poznania" oraz przewodniczący Rincewind Mager wykładowca "Magi transmutacyjnej i Nadprzyrodzonych Gobelinów" a co ważniejsze mistrz Yarreta. Kandydat na uczelnie pamiętając nauki mistrza, nienaganną postawą przywitał się z szanowną komisją przedstawiając się dokładnie. Egzamin Yarreta trwał dłużej niż poprzednich kandydatów. Pytania jakie zadawali egzaminatorzy oprócz pytań o magię i czary także pytali o historię, języki, znajomość geografii. Następnie poprosili o pokaz umiejętności magicznych niziołka. Nasz bohater zaprezentował najlepsze co w tamtym momencie miał przygotowane zmienił swój wygląd w Rincewinda (Co mag skomentował: Jestem już taki stary?). Po 40 min komisja naradzając się przyjeła kandydata na pierwszy rok studiów na kierunku Magii Stosowanej.\\

    Żak Yarret Birgim III został wpypuszczony przez drzwi po przeciwnej stronie sali gdzie czekali jego koledzy Żacy Vincent i Coen. Po chwili za niziołkiem wyszła komisja gratulując kondydatom dostania się. Rincewind podczas gratulacji składanych Yarretowi nakazał mu być rano w zakładzie tkackim bo jest duże zamówienie na gobeliny z herbem miasta. Zacy udali się z radości na piwo. Dwa lata studiowania na Akademii pozwoliły Yarretowi rozwinąć się jak nigdy dotąd. Poznawał nowe zaklęcia, uczył się o stworach magicznych i zwyczajnych. Uczono go o sztuce wojny i jak zachowywać się na polu bitwy. Organizowano pojedynki magiczne między żakami. Uruchomił nawet koło tkactwa pod nadzorem Rincewinda który po pierwszym roku przyją go jako swojego podopiecznego. Tam dzięki doświadczeniu w tkaniu nauczył się pleść magiczne "Nadprzyrodzone Gobeliny" i jako temat swojej pracy wybrał właśnie "Historie "Magicznych Gobelinów" i ich zastosowanie w bitwach". Nadzór Rincewinda stał się jeszcze bardziej surowy i wymagający niż w zakładzie. Studiowanie magicznych przedniotów, ich właściwości i rozmpoznawanie stanowiło nie lada problem z początku, ale praktyczne zajęcia pokazały smykałkę do magi praktycznej anieżeli teorii. Tworzenie eliksirów i naukę o ziołach jako fakultet wybrał dobrowolnie z powodów dostępu do ziół. Przed tym właśnie kramem pani Maggabet poznał niezwykłą kobietę. Anastazja to wspaniała niziołczyni. Cud kobieta jak zwał ją Yarret. Właśnie krzyczała na 3 sprzedajczyków kwiatów mówiąc im:\\

    - Jak mnie nie zostawicie w spokoju to wyrwę wam te kwiatki z koszyka i z waszych dup zrobię sobie piękne flakony na kwiaty!\\

    Jedyne co Yarret słyszał to:\\

    -Jak... ...te kwiatki... ...piękne.\\

    Zauroczony postanowił szybko zaimponować i bardzo szybko wyszedł i używając magii przegonił natrętnych kupczyków. Zaraz potem zaprosił Anastazję na kolację i tak poznał swoją żonę. Ale do Anastazjii jeszcze wrócimy, wcześniej na uczelni miał sporo dziewczyn, niestety zazwyczaj jego związki się rozpadały ze względu na odległość (1 metr wzrostu). Po obronie i otrzymaniu tytułu bakałarza postanowił pozostać na uczelni i zdobyć tytuł magistra. Oczywistym jest że w tym czasie odwiedzał rodzinę i znajomych (teraz już przyjaciół) Coena i Vincenta (Vincent został jego pierwszym na ślubie za co Coen strzelił trochę focha ale na weselichu mu przeszło). Obydwaj przyjaciele pod czujnym okiem kolegi pracowali nad tkaniem w założonym wcześniej klubie gdzie potem zostali vice-prezesami klubu. A Yarret chłoną więdzę jak się koncentrować podczas rzucania czarów i jak można używać krwi aby czary wzmocnić. Szczególnie przydatny był tu trening jaki mieli we trzech poza zajęciami, kiedy na placu często rywalizowali kto kiedy i co zrobi szybciej lepiej i więcej. Wyniki bywały różne. W tym momencie wybrał swą specjalizację transmutacji rezygnując z ewokacji i enchantmentu oraz począ badać smoczą magię. W tym celu jeździł po całym Thay i zbierał próbki smoczych łusek, kości, zębów i innych organów. Miał nawet styczność zetknąć się z żywymi smokowatymi a nawet obserwować prawdziwego smoka. Bardzo pomocna okazała się już wtedy jego nażeczona gdyż znała i studiowała historię Fearunu co pozwoliło zidentyfikować gdzie odbywały się bitwy z udziałem tych stworów i taj magii.\\

    Na trzecim roku magistratu Yarret z Anastazją wzięli ślub. Na weselu zjawiły się obie rodziny... Impreza trwała tydzień w rodzinnym mieście Anastazjii a gości było około 800. Dzieki pracy w zakładzie tkackim przez cały ten czas i wspólnymi oszczędnościami kupili sobie nawet niewielki domek nieopodal akademii. (Dwie sypialnie, kuchnia, łazienka, schowek i maly kramik który przerobiony został na pracownię Yarreta, około 8k golda). Rok póżniej ze specjalizacją w magii Transmutacji, pracą pod tytułem "Smocza magia i jej uniwersalne zastosowanie" Yarret został magistrem Akademii Bezanthur. Rincewind zadowolony z postępów i badań podopiecznego postanowił wysłać Yarreta poza Thay aby rozpocząć zbieranie i badanie tematu związanego z długowiecznością, nieśmiertelnością i wieczną młodością. Niziołek zainteresowany tymi tematami gdyż widział co się dzieje po bitwach chciał mieć pewność, że nie zginie i nie zostawi brzemiennej żony samej. Po rozmowie z Rincewindem otrzymał rekomendacje od Thay na badania i udzielenie pomocy od pewnej gildii.\\



    \end{document}
