\documentclass{article}

\usepackage{a4wide}
\usepackage{polski}
\usepackage[utf8]{inputenc}

\title{\Huge{Szturm na dom Barona}}
\author{Arkadiusz Popczak}
\begin{document}
\maketitle

Po pokonaniu Szepczących z pomocą szefa komórki Kerla drużyna udała się do kanałów aby tam znaleźć sposób na dostanie się do domu barona. \\
\paragraph{\Large{Baron}}\mbox{}\\

\indent
Baron to potężnie zbudowany człowiek. Metr-osiemdziesiąt wzrostu, 90kg mięśni i umiejętności walki szablą jak niejedenwprawny wojownik. Oprócz tego jego ważnym atutem jest spryt i kontrolowanie innch. Szturm jaki się dokonał jest ostatnią deską ratunku dla szlachciców z miasta. Jeżeli się nie uda zarówno Nameless guild oraz straż i reszta możnych odejdzie do lamusa a władzę nad miastem przejmie Baron. Dodatkowym atutem przypuszczenia szturmu była informacja o znalezieniu pierścienia jednego z kuzynów w mięsie dawanym przez Barona.

\begin{table}[h]
\caption{Statystyki Barona}
\centering
    \begin{tabular}{|c|c|c|c|c|c|c|c|}
        \hline
        WW & US & K & Odp & Zr & Int & SW & Ogł \\ \hline
         &  &  &  &  &  &  &  \\ \hline
        A & Żyw & S & Wt & Sz & Mag & PO & PP \\ \hline
         &  &  &  &  &  &  &  \\ \hline
        \end{tabular}
\end{table}
\noindent
Umiejętności:
\hspace{150pt}
Zdolności:\\
\vspace{50pt}\\
Ekwipunek:\\
\vspace{50pt}

W środku nie powinno dojść do otwartej walki gdyż przewagą liczebną oraz siłową BN pokonają BG. Dlatego pozostaje podejście rozmowy, skradania lub inne podejście taktyczne, pozwalające np. podpalić budynek od środka. SZCZEGÓLNIE ODRADZAĆ WALKĘ. Możliwymi opcjami będzie dla przykładu znalezienie Szepczącego który otacza budenek zaklęciem nie pozwalającym go podpalić przez co umożliwią dostanie się do środka oddziałom straży i wojska. Dom jest naprawdę duży. Składający się z piwnicy, partera i dwóch pięter budynek jest dobrze wyposażony i ufortyfikowany. W każdej okiennicy jest otwór strzelniczy dla broniących się. W środku jest wiele pomieszczeń pozwalających wykonać wąskie gardło jak i wiele skrytek w których można ustawić płapki. Każde z pięter przedstawione jest w załączonych mapach. 

\paragraph{\Large{Piwnica}}\mbox{}\\

Piwnice. Mokre, śmierdzące zgnilizną i padliną pomieszczenie ma 10 cel w których trzymane są towary i niektóre kosztowności właścicela. Tutaj też przetrzymywani są więźniowie. Cztery pomieszczenia przydzielone są na zapasy jedzenia i poidła (w tym alkoholu), kolejne trzy na więźniów, dwa na kosztowności a ostatnie, zamknięte przez drzwi ze stalowymi okuciami i 4 zamkami na specjelny żywy towar. W pomieszczeniach przeznaczonych na spiżarnie znajduje się sporo ludzkiego mięsa, oprócz tego jest tam skład soli i innych już pospolitych przypraw, oraz składy na popularne warzywa i owoce jak ziemniaki, marchew, por, jabłka czy cebula. Te składy nie są zamknięte na kłódki. Trzy kolejne pomieszczenia zajmowane są poprzez więźniów. Jest to dwóch zwykłych obywateli którzy w jakiś sposób mieli się przydać Baronowi (możliwe, że specjalne przypadki dla Szepczących). W ostatniej celi znajduje się martwy kiedyś dobrze ubrany lecz storturowany szlachcic, który wyjechał jakiś czas temu po środki do rodziny. W celach przeznaczonych na kosztowności można znaleźć skrzynie ze srebrną i złotą zastawą, stare monety, zwykłe monety, kamienie szlachetne itd. Drzwi do ostatniego pomieszczenia są bardzo grube, przybite wielkimi stalowymi gwoźdźmi szyny metalu trzymają je przy ścianie. W tych drzwiach nie ma okienka. Szyny te dodatkowo zamknięte są na 4 kłódki wielkości niziołczej głowy. Jeżeli BG zdecydują się otworzyć drzwi za nimi będzie czakać minotaur. Do piwnicy można dostać się na dwa sposoby, wybić dziurę z kanałów, lub przez drzwi. (W pomieszczeniu z ludzkim mięsem jest specjalna klapa, dzięki której można uciec z miasta niepostrzeżenie.)

\paragraph{\Large{Parter}}\mbox{}\\

Na parterze znajduje się pięć pomieszczeń oraz korytarz dzięki któremu można dostać się na wyższe piętra lub do piwnicy, a także łączy on wszystkie pomieszczenia. To tutaj jest najwięcej zabarykadowanych okien i drzwi. W każdym pomieszczeniu teraz organizowany jest mały oddział przydzielony do obrony (4-5 BN) piętrem i obroną kieruje stary oddźwierny, który także jest cholernie zręcznym atletą walczącym broniami naręcznymi (traktować jak broń jednoręczną). Jest to jedyna postać, dzięki której BG mogą zostać zauważeni. 

\begin{table}[h]
    \caption{Statystyki starego oddźwiernego}
    \centering
        \begin{tabular}{|c|c|c|c|c|c|c|c|}
            \hline
            WW & US & K & Odp & Zr & Int & SW & Ogł \\ \hline
             &  &  &  &  &  &  &  \\ \hline
            A & Żyw & S & Wt & Sz & Mag & PO & PP \\ \hline
             &  &  &  &  &  &  &  \\ \hline
            \end{tabular}
\end{table}
\noindent
Umiejętności:
\hspace{150pt}
Zdolności:\\
\vspace{50pt}\\
Ekwipunek:\\
\vspace{50pt}

W pomieszczeniach, oprócz walczących ludzi barona (których jest tyle aby w otwartej walce wygrać z atakującymi gdyby nie BG) znajduje się pólki stoły i sporo książek i zwojów. Każde z tych pomiezczeń służyło do przyjmownia. odprawiania i wydawania pieniędzy pracownikom, klientom czy ludziom zaufanym. Wyjście z piwnicy znajduje się w małym pomieszczeniu 

\pagebreak
\centering
\paragraph{\Large{Appendix}}\mbox{}\\
Statystyki oraz ekwipunek zbirów i najemników Barona.
\begin{table}[h]
    \caption{Zbir 1}
    \centering
        \begin{tabular}{|c|c|c|c|c|c|c|c|}
            \hline
            WW & US & K & Odp & Zr & Int & SW & Ogł \\ \hline
             &  &  &  &  &  &  &  \\ \hline
            A & Żyw & S & Wt & Sz & Mag & PO & PP \\ \hline
             &  &  &  &  &  &  &  \\ \hline
            \end{tabular}
\end{table}

\raggedright

\noindent
Umiejętności:
\hspace{150pt}
Zdolności:\\
\vspace{50pt}
Ekwipunek:
\vspace{50pt}


\end{document}
