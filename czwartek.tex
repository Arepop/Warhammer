\documentclass{article}

\usepackage{a4wide}
\usepackage{polski}
\usepackage[utf8]{inputenc}
\usepackage{float}
\usepackage{warhammer}

\newcommand{\whtable}[1]{
    \begin{table}[H]
        \caption{{#1}}
        \vspace{4pt}
        \centering
            \begin{tabular}{|c|c|c|c|c|c|c|c|}
                \hline
                WW & US & K & Odp & Zr & Int & SW & Ogł \\ \hline
                 &  &  &  &  &  &  &  \\ \hline
                A & Żyw & S & Wt & Sz & Mag & PO & PP \\ \hline
                 &  &  &  &  &  &  &  \\ \hline
                \end{tabular}
    \end{table}

    \noindent
    Umiejętności:
    \hspace{150pt}
    Zdolności:
    \vspace{50pt}\\
    Ekwipunek:
    \vspace{50pt}
}

\newcommand{\paragraphx}[1]{
    \paragraph{\Large{{#1}}}\mbox{}\\
}

\title{\Huge{Szturm na dom Barona}}
\author{Arkadiusz Popczak}
\begin{document}
\maketitle

Po pokonaniu Szepczących z pomocą szefa komórki Kerla drużyna udała się do kanałów aby tam znaleźć sposób na dostanie się do domu barona. \\
\paragraph{\Large{Baron}}\mbox{}\\

\indent
Baron to potężnie zbudowany człowiek. Metr-osiemdziesiąt wzrostu, 90kg mięśni i umiejętności walki szablą jak niejedenwprawny wojownik. Oprócz tego jego ważnym atutem jest spryt i kontrolowanie innch. Szturm jaki się dokonał jest ostatnią deską ratunku dla szlachciców z miasta. Jeżeli się nie uda zarówno Nameless guild oraz straż i reszta możnych odejdzie do lamusa a władzę nad miastem przejmie Baron. Dodatkowym atutem przypuszczenia szturmu była informacja o znalezieniu pierścienia jednego z kuzynów w mięsie dawanym przez Barona.

\whtable{Statystyki Barona}

W środku nie powinno dojść do otwartej walki gdyż przewagą liczebną oraz siłową BN pokonają BG. Dlatego pozostaje podejście rozmowy, skradania lub inne podejście taktyczne, pozwalające np. podpalić budynek od środka. SZCZEGÓLNIE ODRADZAĆ WALKĘ. Możliwymi opcjami będzie dla przykładu znalezienie Szepczącego który otacza budenek zaklęciem nie pozwalającym go podpalić przez co umożliwią dostanie się do środka oddziałom straży i wojska. Dom jest naprawdę duży. Składający się z piwnicy, partera i dwóch pięter budynek jest dobrze wyposażony i ufortyfikowany. W każdej okiennicy jest otwór strzelniczy dla broniących się. W środku jest wiele pomieszczeń pozwalających wykonać wąskie gardło jak i wiele skrytek w których można ustawić płapki. Każde z pięter przedstawione jest w załączonych mapach. Na każde piętro może zostać napuszczony Minotaur.
\pagebreak
\paragraph{\Large{Piwnica}}\mbox{}\\

Piwnice. Mokre, śmierdzące zgnilizną i padliną pomieszczenie ma 10 cel w których trzymane są towary i niektóre kosztowności właścicela. Tutaj też przetrzymywani są więźniowie. Cztery pomieszczenia przydzielone są na zapasy jedzenia i poidła (w tym alkoholu), kolejne trzy na więźniów, dwa na kosztowności a ostatnie, zamknięte przez drzwi ze stalowymi okuciami i 4 zamkami na specjelny żywy towar. W pomieszczeniach przeznaczonych na spiżarnie znajduje się sporo ludzkiego mięsa, oprócz tego jest tam skład soli i innych już pospolitych przypraw, oraz składy na popularne warzywa i owoce jak ziemniaki, marchew, por, jabłka czy cebula. Te składy nie są zamknięte na kłódki. Trzy kolejne pomieszczenia zajmowane są poprzez więźniów. Jest to dwóch zwykłych obywateli którzy w jakiś sposób mieli się przydać Baronowi (możliwe, że specjalne przypadki dla Szepczących). W ostatniej celi znajduje się martwy kiedyś dobrze ubrany lecz storturowany szlachcic, który wyjechał jakiś czas temu po środki do rodziny. W celach przeznaczonych na kosztowności można znaleźć skrzynie ze srebrną i złotą zastawą, stare monety, zwykłe monety, kamienie szlachetne itd. Drzwi do ostatniego pomieszczenia są bardzo grube, przybite wielkimi stalowymi gwoźdźmi szyny metalu trzymają je przy ścianie. W tych drzwiach nie ma okienka. Szyny te dodatkowo zamknięte są na 4 kłódki wielkości niziołczej głowy. Jeżeli BG zdecydują się otworzyć drzwi za nimi będzie czakać minotaur. 

\whtable{Minotaur}

Do piwnicy można dostać się na dwa sposoby, wybić dziurę z kanałów, lub przez drzwi. (W pomieszczeniu z ludzkim mięsem jest specjalna klapa, dzięki której można uciec z miasta niepostrzeżenie.)

\paragraph{\Large{Parter}}\mbox{}\\

Na parterze znajduje się pięć pomieszczeń oraz korytarz dzięki któremu można dostać się na wyższe piętra lub do piwnicy, a także łączy on wszystkie pomieszczenia. To tutaj jest najwięcej zabarykadowanych okien i drzwi. W każdym pomieszczeniu teraz organizowany jest mały oddział przydzielony do obrony (4-5 BN) piętrem i obroną kieruje stary oddźwierny, który także jest cholernie zręcznym atletą walczącym broniami naręcznymi (traktować jak broń jednoręczną). Jest to jedyna postać, dzięki której BG mogą zostać zauważeni. 


\whtable{Statystyki starego oddźwiernego}

W pomieszczeniach, oprócz walczących ludzi barona (których jest tyle aby w otwartej walce wygrać z atakującymi gdyby nie BG) znajduje się pólki stoły i sporo książek i zwojów. Każde z tych pomiezczeń służyło do przyjmownia. odprawiania i wydawania pieniędzy pracownikom, klientom czy ludziom zaufanym jak i służbie. Wyjście z piwnicy znajduje się w małym, ciemnym pomieszczeniu. Jeżeli dobrze to rozegrają, będzie dało uniknąć się walki (np. przekupstwem, przegadaniem, czy obietnicami pomocy po wszystkim lub skradaniem i przemknięciem.) gdyż obrońcy po nocnych walkach będą też trochę zmęczeni. Jeżeli dojdzie do walki będzie możliwość napuszczenia minotaura na obrońców lub też sprowadzenie walki do piwnicy (przewaga liczebna nie będzie mieć aż takiego znaczenia, ale zmęczenie BG owszem), lub też pomoc od więźniów w odwróceniu uwagi. Po wydostaniu się z pomieszczenia na korytarz, będzie można przemknąć na pierwsze oraz drugie piętro. W zależności od podjętej decyzji baron wraz z dwoma pomocnikami ucieknie (jeżeli będą chcieć badać pierwsze piętro) lub też czeka ich walka jeżeli udadzą się od razu na drugie piętro. 

\paragraph{\Large{Pierwsze piętro}}\mbox{}\\

Pierwsze piętro przeznaczone było dla pomocników i wspólników, którzy stali znacznie wyżej niż zwykli pracownicy. To tutaj swoje pomieszczenie obrad mieli Szepczący oraz to na tym piętrze przechowywane są mniej ważne informacjie o rodzinie Barona. Prowadzone statystyki co do zatrudnionych i zabitych. Sypialnie Szepczących oraz biblioteki. Teraz tutaj znajduje się główny skład broni i zapasów któe służą ludziom do obrony przeciwko miastu. Piętro składa się z 6 pokoii. Dwa z nich są to sypielnie dla Szepczących, jeden większy to sala obrad, kolejny to jadalnia, duża i obszerna biblioteka oraz biuro do przyjmowania petentów. W biurze przesiaduje dwóch Szepczących, którzy podtrzymują zaklęcie obrony budynku przed pociskami i ogniem. 

\whtable{Szepczący}

Jeżeli gracze zajmą się Szepczącymi to Baron ucieknie i drugie piętro będzie mniejszym wyzwaniem jak i mniejszymi narodami. W pokojach można znaleźć ciekawe księgi czy dokumenty i traktaty handlowe jak i umowy z mieszkańcami i szlachcicami z innych miast. W każdym pokoju oprócz znajdziek są obrońcy po 4 na pokój. 

\paragraph{\Large{Drugie piętro}}\mbox{}\\

Drugie piętro składa się z osobistych kwater Barona. Jedna wielka sala, łazienka, prywatna jadalnia. W sali jest ogromne biurko, mnóstwo półek i szafek. Bogato zrobiona w złoto i kaszmir. Piękne skórzane koucia foteli. Przepych pełną parą. Obrazy, portrety, mapy świata w pięknych zdobionych klejnotami ramach. Niesamowite bronie ozdobne na ścianach. W jadalni podobnie. Złota zastawa wysadzana rubinami. Sztućce z białego złota idt. itp. Sam Baron siedząc w fotelu (jeżeli gracze pobiegli tam od razu) czeka na BG w fotelu za biurkiem gdzie celuje muszkietem w drzwi wraz z dwoma synami. Z Baronem będzie można porozmawiać i usłyszeć od niego swoje racje. Opowie o głodzie panującym w mieście kilka lat temu. O rozbestwieniu strażników jak i o jego planie na skorumpowanie szlachty. Opowie jak zabito jego żonę i kto to zrobił. Wykorzystanie sytuacji aby zmniejszyć populację. opowie o odwiedzinach jakiegoś maga czy czarnoksiężnika, który zaproponował mu pomoc szepczących w budowaniu armii w zamian za kilka eksperymentów. Opowie o problemach i tego co musiał się podjąć przed upadkiem. Opowie dlaczego on jest bogaty a inni jedzą ludzkie mięso. A potem albo zacznie się walka, albo gracze pozwolą mu uciec, albo Baron przekona graczy albo gracze Barona. Jeżeli gracze nie zastaną barona na biurku będzie można znaleźć list do szlachty i władz mówiący o zemście i załączone kilka dokumentów obarczających szlachtę za niektóre sytuacje w mieście, jak i o wpływach oficera War district sir Hansa Krugera i koneksjach z nim samym.



\pagebreak
\centering
\paragraphx{Appendix}
Statystyki oraz ekwipunek zbirów i najemników Barona.
\raggedright

\startwhtable[Zbir]
\primary 37|27|34|25|32|34|36|24
\secondary 1|12|3|2|4|0|0|0
\stopwhtable

\startwhtable[Zbir 2]
\primary 27|37|24|35|32|34|36|24
\secondary 1|12|2|3|4|0|0|0
\stopwhtable

\startwhtable[Zbir 3]
\primary 42|32|39|30|32|34|36|24
\secondary 1|14|3|3|4|0|0|0
\stopwhtable

\startwhtable[Zbir 4]
\primary 32|42|30|39|32|34|36|24
\secondary 1|14|3|3|4|0|0|0
\stopwhtable


\end{document}
