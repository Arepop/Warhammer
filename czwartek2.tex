\documentclass{article}

\usepackage{a4wide}
\usepackage{polski}
\usepackage[utf8]{inputenc}
\usepackage{float}
\usepackage{warhammer}
\usepackage[ampersand]{easylist}

\title{\Huge{Śledstwo w sprawie korupcji}}
\author{Arkadiusz Popczak}
\begin{document}
\maketitle

Po szturmie na doma Barona i jego ucieczce wraz z synami ostatnie co chciał zrobić było pogrążenie straży miejskiej, a właściwie głównego oficera Hansa Krugera, co mogłoby pozostawić szlachtę bez obrony a tym samym dać mu szansę na powrót do miasta. Sam Baron udał się do swoich współpracowników i pracodawców aby wyposażyć go w środki ludzkie i pieniężne aby w odpowiednim momencie uderzyć na miasto. Niestety poprzez przykre wydarzenia podczas szturmu na dom Barona, Max, niziołek, kopnął w kalendarz, przez przypadek, z powodu podania mu zepsutej mikstury przez Alfreda. Do Wilhelma oraz Alfreda dołącza Felix Grand. Sierżant straży miejskiej. Razem z nimi mają wspólną misję zinflitrować szlachtę, zbdać jak jest skorumpowana oraz co się dzieje teraz w straży i slumsach.

\startwhtable[Hans Kruger]
\primary 67|47|54|45|32|34|56|44
\secondary 3|29|5|4|4|0|0|2
\stopwhtable

\paragraphx{Możliwości i ograniczenia}
Od teraz do dyspozycji graczy jest 2 podwładnych na głowę. Odpowienio ze swoich organizacji. Są to podstawowe postacie na początku rozwoju ze statystykami bandziora (partz Appendix). Mogą ich wysyłać na misje oraz rządać przekazania informacji czy zbadania prostej rzeczy. Muszą brać pod uwagę, że przy trudniejszych misjach BN mogą zginąć. Dodatkowo Ci podwładni nie rozwijają się. Ograniczenia finansowe to około 1000zk na miesiąc na głowę dla gildii i około 300zk funduszu operacyjnego na miesiąc dla sierżanta. Oprócz ograniczeń w ludziach oraz finansowych BG ogranicza nadal zwierzchnictwo.\\
Lista zaufanych:
\begin{enumerate}
    \item Rose Johnson (h)
    \item Gerfried Lippert (h)
    \item Gangolf Berlepsch (h)
    \item Hans Kruger (h)
    \item Ida Feig (d)
    \item Drirulugg Metalgrog (d)
    \item Srakurer Steelrage (d)
    \item Bel-Zaecrurhac The Glorious (e)
    \item Lifanrevi The Fearless (e)
    \item Gilly Bracend-Pinster (n)
\end{enumerate}

\paragraphx{Aktualna misja i cele gildii}
Gildia działa w ukryciu chcąc obalić rządy szlachty i skorumpowanie w imperium. Dlatego podburza ludzi i ściąga przeciw sobie największych wpływowych ludzi w mieście. Głownym celem jest rozpoczęcie pospolitego ruszenia w mniejszych miastach aby wprowadzić tam zaczątki demokracji oraz zrównać szlachtę i ich bogatctwa do poziomu mieszczan. Aby to zrobić na początku trzeba przejąć władzę i zmienić prawo w miastach. Po tych wydarzeniach miastem będzie rządzić rada wybrana przez pospólstwo mieszczan oraz szlachtę. (Przywileje szlachty zostaną odebrane a prawo wyboru zdobędą wszyscy). 

\paragraphx{Dzielnica wojskowa}
W dzielnicy wojskowej (nazwanej tak po tym jak w mieście kiedyś stacjonowała armia) przebywa teraz straż miejska oraz organizowane są tam zawody sportowe jak i krwawe. Znajduje się tam więzienie i budynki strażników. Przywódcą straży jest \textbf{Generał Bel-Venezel The Enforcer (e)} a pod nim znajduje się czterech oficerów dowodzących. Jednym z nich jest \textbf{Hans Kruger}. Posostała trójka to krasnal \textbf{Ekruc Strongforged} i krasnoludzica \textbf{Dirgruh Leadhand} oraz elf \textbf{Thurlocun Scaca}. Każdy z nich ma kilku sierżantów którym może wydawać rozkazy. Straż składa się z około 300 osób. Hans Kruger jest skorumpowanym oficerem który działając dla gildii pomógł podczas szturmu na Barona i w odzyskaniu slumsów dla miasta. W tym momencie jego zadanie to osłabić straż na tyle aby gildia mogła zbrojnie zająć dzielnice szlachecką i przejeła do czasu wyborów władzę w mieście.

\paragraphx{Szlacheckie niesmacki}
Aby szlachta nie mogła się bronić a gildia miała dobre argumenty do wykopania szlachty BG podejmą się śledstwa które pokaże powiązania szlachty z półświatkiem oraz o skorumpowaniu i złym traktowaniu mieszkańców (opisy dzielnic patrz. Appendix 2).

\paragraphx{Szczegóły misji}
Śledstwo zacznie się od zapoznania się z obyczajami szlachty. Gdzie się spotykają, kto zasiada w radzie jak do tej rady się dostać. 

\paragraph{Obyczaje szlachty i spotkania}\mbox{}\\\indent
Zadanie to można rozwiązać na wiele sposobów. Szpiegostwo, wypytywanie ludzi, wciąnięcie się w szlacheckie lobby czy przyłączenie się do gry w remika. Kiedy gracze zdecydują w jaki sposób chcą wykonać tą część zadania należy przedstawić im opis obyczajów szlachty. Szlacheckie życie o bardzo lekkie życie. Każdy z nich codziennie chodzi do restauraci zajmuje się sztuką czy uczy fechtunku. Raz na tydzień organizowana jest wielka fiesta, która kosztuje 200zk od głowy oraz tytuł szlachecki aby tam w ogóle wejść. Szlachcianki bardzo często oddają się odpoczynkowi na łone natury w parku lub chodzą do biblioteki. Nie stronią też od kobiecych gier czy zabaw jakie miały miejsce w średniowieczu. Szlachcianie zajmują się hazardem oraz interesami. Do hazardu można się przyłączyć wpłacając odpowiednią ilość złota (100zk) dla banku. Złoto to pokrywa koszty trunków oraz jadła. Ponadto są stoły otwarte i zamknięte. Aby dostać się do stołów zamkniętych należy mieć już pewną reputację. Każdy ze szlachty mieszka w swoich dworach za miastem. W mieście przebywa w większości Rada miasta. Dodatkowo, każdy ród ma swoich prywatnych ochroniarzy, którzy oddadzą za nich życie. 

\paragraph{Rada miasta}\mbox{}\\\indent
Miastem zarządza rada miejska składająca się z 10 najbardziej wpływowych szlachciców. Są to właściciele ziemscy i leśni, zarządcy browarów, tutejsza głowa banku zarówno jednego jak i drugiego, szef administracji, właściciel knajpy w dzielnicy szlacheckiej, zarządca zielenią. To oni ustalają na co wydawać środki z podatków i co w mieście się zmienia. Ich największym problemem był Baron, który teraz został rozwiązany. Na aktualny moment w radzie miasta jest spokój a radni już planują jak pozbyć się slumsów i pola namiotowego aby tamtejszy teren zagospodarować na nową dzielincę rozrywek. Rada na co dzień stacjonuje w mieście ale każdy z nich także ma swoje dwory i dworki za miastem gdzie chętniej przebywają. W radzie miasta znajduje się tekże jedna kobieta która jest głową największego rodu w okolicy miasta.\\

\begin{easylist}[enumerate]
& \textbf{Karl Mittag} (h) Właściciel połowy lasów
&& Nie ma ziemksich upraw ale na sprzedaży i wysyłce drewna dorobił się majątku. Teraz wykupione ma ponad połowę ziemi leśnej a biznesem zajmują się jego pracownicy.
&& Jest jedną z osób które sprzedają drewno za granicę w dużo lepszej cenie. Wyprzedając w ten sposów możliwość rozwoju technologii i armii imperium. Są podejrzenia że współpracuje z wrogim krajem (A tak naprawdę z wojownikami chaosu z za gór)
& \textbf{Meinrad Tillich} (h) Właściciel spólki "Sail n sound"
&& Współpracuje z 1, 10 i 5 mają podpisane umowy na wysyłkę eksport i import towarów. Jego spółka jest najbezpieczniejsza w okolicy ale i najdroższa. Nie przewozi nic niżej niż równowartość 4000zk.
&& Często na statkach przewożeni są niewolnicy oraz cenne dokumenty dla zarządców prowincjami które "czasami są podrabiane.
& \textbf{Volkmar Kantorowicz} (h) Właściciel browaru
&& Właściciel browaru. Sprzedaż piwa w samym mieście i do prywatnych piwnic szlachty i mieszczan. Dodaje środki uzależniające do zwiększenia sprzedaży.
&& Środki te oprócz uzależnienia niszczą dodatkowo organizmy ludzi i niziołków co powoduje bezdzietność i bezpłodność. Co przekłada się na brak przyrostu naturalnego.
& \textbf{Irmgard Hänel} (h) Największy i najbogatszy ród
&& Kobieta, głowa największego rodu w mieście. Pieniądze ostały jej się z wypraw wojennych. Po godzinach zajmuje się magią i alchemią oraz wytwarzaniem złota. 
&& Wprowadzanie własnych bitych monet na rynek niszczy ekonomię miasta na czym cierpią mieszczanie i plebs. Pozatym jest wamirzą lamią a w radzie zasiada jej ludzki dopelganger.
& \textbf{Endok Commonheart} (d) Biznesmann importujący cenne towary z zagranicy
&& Bawełna, skóry rzadkich okazów zwierząr, pióra gryfa czy czarny olej. To wszystko jego działka. 
&& Jedyny który dorabia się na samym sobie. Przekupuje straż miejską materiałami na broń i zbroje. Chce demokracji bo pomoże mu to rozkręcić większy biznes.
& \textbf{Krogon Brightkeeper} (d) Właściciel dworku w dzielnicy
&& Zatrudnia najznamienitrzych kucharzy i zapewnia rozrywkę w karczmie.
&& Prowadzi też sieć burdeli. I zajmuje się handlem żywym towarem. 
& \textbf{Nogait Stormmore} (d) Głowa banków
&& Oszustwa podatkowe, i zdobyczne pieniądze z depozytów jak ktoś "znika". Oprócz tego inwestycje w waluty.
&& Morderstwa mieszczan, kradzież pieniędzy i kredyty których nie da się spłacić i trzeba to zrobić na inne sposoby. Przemyca narkotyki do miasta.
& \textbf{Zucrae Khaerla} (e) Szef administracji miasta
&& Zarabia na rozwoju miasta. Większość podatków za mieszkania, ziemie czy prowadzenie karczm i staji idzie do niego. 
&& Co miesięcznie podnosi podatki wmawiając ludziom że to na zbrojenie i podniesienie bezpieczeństwa miasta.
& \textbf{Zizkon Tornon} (e) Najwyższy sędzia.
&&  Zajmuje się egzekwowaniem prawa w mieście
&& Łapówki i uprowadzenia świadków, fałszowanie zeznań i pozbywanie się kogo trzeba. (Jak konkurencja na rynku i zapłacenie kary aby ją stłamsić.)
& \textbf{Jaquob Sarazin} (hf) 70$\%$ upraw jest jego.
&& To od niego jest większość wieśniaków pracujących na roli. Potrafi dostarczyć jedzenia dla połowy miasta
&& Na jego ziemiach ludzie głodują bo mają obiecane, że na zimę będzie im wszystko oddane z nawiązką. Nie przejmuje się chorobami a w razie śmierci plebsu po prostu sprowadza nowy.
\end{easylist}\mbox{}\\


\paragraphx{Tam gdzie boli najbardziej}

Do każdego z rady będzie oddzielna misja która będzie dawała możliwość odkrycia kart danego szlachcica. Aby dało w mieście się zainterweniować BG muszą obarczyć winą co najmniej 6 szlachciców oraz osłabić straż aby zbrojnie nie przejeła władzy w mieście.

\paragraph{Karl Mittag}\mbox{}\\\indent
Aby to sprawdzić BG powinni wybadać na czym zarabia szlachcic a następnie dowiedzieć się, że drewno w pewnym momencie transportu znika. Po zbadaniu tego jeżeli udadzą się na miejsca wycinki i transportu, które znajduje się głębiej w lesie a drwale mają tam swoje siedlisko, wybadają, że drewno pakowane jest na łodzie gdzie nikt nie wie z pracujących co się dalej z nimi dzieje. Aby tego się dowiedzieć będą musieli dostać się na łódź i popłynąć nią. Zanim to się stanie, w lesie będzie można spotkać niespokojne i niezadowolone z prac duchy lasu i wrogów mieszkających w lasach (patrz bestiariusz). Po dopłynięciu na miejsce przekazania ładunku, gracze zobaczą wojowników chaosu płacących złotem za drewno. Aby sprowadzić winę na Karla trzeba będzie albo przekonać kapitana statku do spowiedzi przed graczami albo dojdzie do walki z chaosem i głowa wojownika oraz drewno plus zeznania rębaczy.

\paragraph{Meinrad Tillich}\mbox{}\\\indent
Jego przewożenie niewolników i fałszowanie dokumentów jest bardzo subtelne. Wie o tym tylko kilka osób oraz osoby przewożone. Informacje o tym co robi Meinrad dostaną jeżeli pójdą do burdelu lub trafią na bezdomnego. Wypytywanie się o co kolwiek szlachty nic im nie da. Mogą się o tym dowiedzieć także wykonując misję Krogon Brightkeeper. Kiedy będą wiedzieć co robi Meinrad aby obciążyć go zarzutami i mieć na niego haka trzeba będzie dostać się lub nasłać straż (korupcja straży) na statek przewożący towar i trzeba będzie dowiedzieć się w którym transporcie przewożone będą dokumenty do podrobienia lub niewolnicy. Na statku pod pokładem zamkniętych będzie 10-12 kobiet dzieci i mężczyzn w nędznym stanie. W kajucie kapitana zaś będą narzędzie służące do fałszowania. BG będą mogli przegadać kapitana, przekupić go, zmusić siłą do zeznań. Mogą też uwolnić niewolników i przyprowadzić ich w bezpieczne miejsce oraz zabrać dokumenty od kapitana i przekonać nadawców do sprawdzenia treści po tym jak upewnią się że były podrabiane. Mogą też ukraść sprzęt jeżeli na tej podstawie uda im się pokazać winę.







\pagebreak
\centering
\paragraphx{Appendix}
Statystyki oraz ekwipunek zbirów i najemników Barona.
\raggedright

\startwhtable[Zbir]
\primary 37|27|34|25|32|34|36|24
\secondary 1|12|3|2|4|0|0|0
\stopwhtable

\startwhtable[Zbir 2]
\primary 27|37|24|35|32|34|36|24
\secondary 1|12|2|3|4|0|0|0
\stopwhtable

\startwhtable[Zbir 3]
\primary 42|32|39|30|32|34|36|24
\secondary 1|14|3|3|4|0|0|0
\stopwhtable

\startwhtable[Zbir 4]
\primary 32|42|30|39|32|34|36|24
\secondary 1|14|3|3|4|0|0|0
\stopwhtable

\end{document}