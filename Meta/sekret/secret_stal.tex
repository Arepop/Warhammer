\documentclass{article}

    \usepackage{a4wide}
    \usepackage{polski}
    \usepackage[utf8]{inputenc}
    \usepackage{float}
    \usepackage[greek]{babel}
    % \usepackage[LGR]{fontenc}

    \newcommand{\paragraphx}[1]{
        \paragraph{\Large{{#1}}}\mbox{}

    }

    \title{\Huge{Stal}}
    \date{}
    \begin{document}
    \maketitle

    \paragraphx{Przepis}
    Stal – stop żelaza z węglem, plastycznie obrobiony i obrabialny cieplnie, o zawartości węgla nieprzekraczającej 2,11$\%$, co odpowiada granicznej rozpuszczalności węgla w żelazie (dla stali stopowych zawartość węgla może być dużo wyższa). Węgiel w stali najczęściej występuje w postaci perlitu płytkowego. Niekiedy jednak, szczególnie przy większych zawartościach węgla, cementyt występuje w postaci kulkowej w otoczeniu ziaren ferrytu.

    Im większa zawartość węgla, a w konsekwencji udział twardego i kruchego cementytu, tym większa twardość stali. Węgiel w stalach niskostopowych wpływa na twardość poprzez wpływ na hartowność stali; im większa zawartość węgla tym dłuższy czas jest potrzebny do przemiany perlitycznej – co w konsekwencji prowadzi do przemiany bainitycznej i martenzytycznej. W stalach stopowych wpływ węgla na twardość jest również spowodowany tendencją niektórych metali, głównie chromu, do tworzenia związków z węglem – głównie węglików o bardzo wysokiej twardości.
    
    Materiał zawierający (masowo) więcej żelaza niż jakiegokolwiek innego pierwiastka, o zawartości węgla w zasadzie mniejszej niż 2$\%$ i zawierający inne pierwiastki. Ograniczona liczba stali chromowych może zawierać więcej niż 2$\%$ C, lecz 2$\%$ jest ogólnie przyjętą wartością odróżniającą stal od żeliwa.
    
    Stal obok żelaza i węgla zawiera zwykle również inne składniki. Do pożądanych składników stopowych zalicza się głównie metale, zwykle chrom, nikiel, mangan, wolfram, miedź, molibden, tytan. Pierwiastki takie jak tlen, azot, siarka oraz wtrącenia niemetaliczne, głównie tlenków siarki i fosforu zwane są zanieczyszczeniami.

    \end{document}
    