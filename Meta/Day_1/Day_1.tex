\documentclass{article}

    \usepackage{a4wide}
    \usepackage{polski}
    \usepackage[utf8]{inputenc}
    \usepackage{float}
    \usepackage{xcolor}
    \usepackage[ampersand]{easylist}
    \usepackage{tcolorbox}
    \usepackage{graphicx}
    \usepackage{transparent}

    % \usepackage[greek]{babel}
    % \usepackage[LGR]{fontenc}

    \newcommand{\paragraphx}[1]{
        \paragraph{\Large{{#1}}}\mbox{}

    }

    \newcommand{\textbb}[1]{
        \smallskip
        \textbf{{#1}}
        \smallskip
    }

    \title{\Huge{Day 1}}
    \author{Arkadiusz Popczak}
    \date{}
    \begin{document}
    \maketitle

    \section{Przedmowa}
    \textbb{Rok 1231 kalendarza Rodgera I:}

    W wiosce przybywa coraz więcej ludzi. Dziwne stalowe istoty przestały dostarczać nam jedzenia i wody pod żelazną bramą. Dostajemy w nich narzędzia oraz rzeczy z księgami, które rozumieją tylko starsi. Uczą nas jak z nich korzystać. Jeżeli nie będziemy potrzebować stalowych istot aby przetrwać może będziemy mogli utrzymać przyrozt populacji.

    \textbb{Rok 1239 kalendarza Rodgera I:}

    Nasze uprawy i hodowle zwierząt przynoszą efekty. Populacja nasza zwiększyła się dwukrotnie choć wiele dzieci nadal ginie przy porodzie. Starszyzna dostała dziwne rysunki od stalowych. Podobno będziemy teraz budować domy z kamienia.

    \textbb{Rok 1265 kalendarza Rodgera I:}

    Starszyzna została zabita podczas buntu. Piszę więc swoje ostatnie słowa aby przekazać informacje o urządzeniach które zostały do tego użyte. Schematy oraz egzemplarze jakie dostaliśmy zostały zniszczone. Jeżeli ta broń zagłady zostanie rozpowszechniona przyszłość miasta jest dla mnie zagadką.

    \textbb{Rok 1267 kalendarza Rodgera I:}

    Czterysta trzydziesty piąty dzień buntu. Jesteśmy prawie pewni, że tamci zdobyli schematy broni. Wczoraj Andre zobaczył otwartą bramę i stalowych przez nią wchodzących. W nocy słyszeliśmy krzyki po drugiej stronie barykady. Od tamtej pory ataki ustały. Przygotowujemy się na najgorsze choć niektórzy twierdzą, że to my wygraliśmy.

   \textbb{Rok 0 kalendarza po Odkryciu:}

   Odnależliśmy dzisiaj ten dziennik i postanowiłem go kontynuować. Nie mam pojęcia co to za broń ani za stalowe istoty, ale jestem pewien, że zabiły naszych przodków. Nikt nie wrócł żywy spod bramy o której mowa. Jeżeli to ta sama brama którą my napotkaliśmy. Ann oczekuje kolejnego dziecka. Liczba osób: 121.

    \textbb{Rok 30 kalendarza po Odkryciu:}

    Zbliżają się moje ostatnie dni. Osada przeistacza się w coś większego. Mamy dostęp do wody oraz jedzenia. Wszystkie schematy które znalazłem ukryłem aby nie dostały się w niepowołane ręce. Jeżeli mielibyśmy być zmasakrowani tak jak poprzednicy. Jeżeli czytasz ten dziennik wiedz, że nie przynoszą one nic dobrego. Populacja: 430.

    \textbb{Rok 56 kalendarza po Odkryciu:}

    Znalazłem schematy ale nie umiem ich odczytać. dziwne rysunki pomogły nauczyć mi się jak działają niektóre ze znalezisk. Zachowam je na wypadek jakiegoś buntu czy ataku. Pora przejąć władzę nad miastem.

    \textbb{Rok 143 kalendarza po Odkryciu:}

    Zaczyna brakować wody. Staw który zawsze był pełen wysycha. Reszta wody jest zanieczyszczona. Technicy, którzy analizują sytuację nie znają nadal ku temu powodu. Pierwsze wyprawy za stalową barierę nie wracają już od kilku miesięcy. Musimy wysłać kolejną. Jeżeli nie znajdziemy nowego źródła wody czeka nas zagłada. Populacja 1540.

    \section{Problemy z wodą}
    - Wasza 16 została wybrana aby wyjść z naszego wspaniałej osady i iść w nieznane aby zapewnić nam przetrwanie. - Odezwał się do więźniów zarządca osady.

    - Kiedy wyjedziecie miejcie na uwadze, że po powrocie wszystkie wasze grzechy i czyny wobec systemu zostaną wybaczone a wy uznawani będziecie za bohaterów. Wasze życia należą teraz do mnie. Macie jakieś pytania? - Ze zniecierpliwieniem odezwał się zarządca.\bigskip

    \slshape
    W osadzie/mieście "\textbf{Darany}" zaczyna brakować świeżej wody. Od jakichś trzech miesięcy dystrybucja wody została przejęta przez władze. Woda została skażona poprzez wycieki radioaktywnych odpadów z eletrowni, które powstały z powodu usterek w automatycznym uzdatnianiu. Sytuacja ta panuje już sporo czasu i woda w okolicy staje się niezdatna do picia. Elektrownie znajdują się pod ziemią i ciężko do nich jest trafić. Nawet jeśli uda się tam dotrzeć, w tym momencie gracze nie mają odpowiednich umiejętności aby je naprawić.
    W swoją podróż wyruszają pod przymusem jako więzniowie polityczni i jest to ich jedyna szansa na przeżycie jako, że ciąży na nich wyrok śmierci przez powieszenie. Wyposażenie jakie dostają to dwa stare automaty z ograniczoną ilością amunicji. Prowiantem na kilka dni i bronią białą. W wiosce mieszkają ludzie w budynkach pół na pół z cegły i starych metalowych płyt. Technologia wytwarzania metalu dawno była zatracona. Pomimo posiadania schematów broni i instrukcji wytwarzania elemntów metalowych, język jakiego użyto jest dla nikogo teraz nieznany. Aczkolwiek przetrwał w wiosce elementy elektroniki i elektryki z epoki Rodgera, które niektórzy potrafią użytkować (Komputery w stylu IBM PC oraz Compaq Portable). Nikt nie wie jak działają, ani dlaczego. Ale podane tam są koordynaty osady, która powstała obok. Zarówno "Darany" jak i "Wieżowo" powstały jako ośrodki eksperymentalne do testowania ludzkich zachowań przez komputer.

    "Darany" eksperymentowało nad przyrostem naturalnym i stymulowanym rozwojem technologicznym i badały zależność między agresją, gęstością populacji i agresją. Ogrodzony teren byłnieprzekraczalny i każdy kto próbował go opuścić natrafiał na barierę podpiętą do prądu. Gdy zebrano wystarczającą ilość danych osadę podano eksterminacji. Na szczęście kilkanaście osób przeżyło co wystarczyło aby odnowić populację i zasiedlić wioskę. W tym czasie projekt zamknięto. A osadę pozostawiono samą sobie. Bez zwierzchnictwa ludzi komputer nie rozpoczynał nowych eksprymentów i z czasem się zepsuł. Od wtedy można wyjść za płot który trzyma niebezpieczeństwa z dala od osady. (Ci którzy poszli nigdy nie wrócili). \bigskip

    \normalfont
    -Macie jeden dzień na zebranie się i wyruszenie. Odejść.

    \section{Przygotowania}
    \slshape
    Ekwipunek oraz materiały na podróż są już przygotowane. Gracze jedyne co mogą to udać się do starej biblioteki lub do komputera aby dowiedzieć się gdzie się pierwsze udać. Dodatkowo mogą w wybranym miejscu ustalić plan działania. W bibliotece mogą znaleźć mapę osady wraz z zaznaczonym płotem. (Dodatek: Mapa "Darany"). Tą część staramy się skrócić do minimum. Gracze powinni czuć presję na karku aby jak najszybciej wyruszyć z osady. Zdają sobię sprawę z konsekwencji jeśli nie pójdą. Jako osobiste cele najlepiej wybrać jedno z poniższych:
    \bigskip
    \begin{easylist}[itemize]
        & Chęć eksploracji.
        & Ucieczka od śmierci.
        & Badacz lub naukowiec.
        & Zaimponowanie zarządcy.
        & Ratowanie rodziny.
    \end{easylist}

    \bigskip

    Każda z tych opcji może zostać odpowiednio urozmaicona.

    \newpage

    Dla przykładu Ratowanie Rodziny:

    \begin{tcolorbox}
        \ttfamily
        Trzy miesiące temu odcięto swobodny dostęp do wody. Twoja żona będąc w ciąży potrzebiwała jej aby utrzymać stałe nawodnienie. Zdecydowałeś się z paroma kumplami z okolicy, że w nocy pójdziecie pod jeziorko i ukradniecie wodę. Przyłączyli się do Ciebie przeciwnicy zarządcy chcąc pokazać swój bunt jako, że z nimi się zgadzałeś zaplanowaliście to jako symbol walki z niesprawiedliwą władzą. Złapano was kiedy Twój brat upuścił wiadro z wodą na metalowy dach. Szybko was złapano a następnie torturowano. Przez Twoją akcję walki z reżimem i kradzież wody, nie tylko Ty jesteś zagrożony. Nie było to tak dobrym pomysłem na ten moment jak wydawało się wcześniej. Wiesz, że podjęcie się tego zadania zagwarantuje przeżycie nie tylko Tobie, ale również życie Twojej córki i żony, które teraz są w areszcie.
    \end{tcolorbox}


    \bigskip

    Wprawny MG bez problemu wymyśli swoje cele i motywy postaci, lub pozwoli wymyślić je graczom. Dla każdego z wymienionych motywów załączam przykłądowe opisy.

    \bigskip

    Chęć eksploracji:
    \begin{tcolorbox}
        \ttfamily
        Kiedy pierwszy raz wszedłeś do biblioteki wymykając się z domu od razu zainteresowała Cię grubo oprawiona książka. Otworzywszy ją zobaczyłeś jakieś robaczki i szlaczki a pod nimi obrazek dziwnego stwora. Cztery odnóża, głowa i włosy na całym ciele. Podpis pod obrazkiem był "Kot". Chciałeś dowiedzieć się więcej o tej kreaturze. Powoli zacząłeś uczyć się czytać i pisać a w głowie byłeś na wspaniałej przygodzie po dżungli. Po latach pełniąc urząd w składzie zarządcy przeczytałeś nie ten list co trzeba. Złapany przez strażników kiedy próbowałeś spalić informacje trafiłeś do pierdla. Kiedy zaproponowano Ci wyjsć z osady wiedziałeś, że to okazja spełnić marzenia z dzieciństwa. Ale aby podzielić się w przyszłości zdobytą wiedzą koniecznie musisz tu wrócić.
    \end{tcolorbox}


    \bigskip

    Ucieczka przed śmiercą:

    \begin{tcolorbox}
        \ttfamily
        Cztery lata prac przy kamieniołomie dały Ci czas na refleksje i przemyślenia. Dwie rodziny, które wymordowałeś zanim zostałeś złapany do dzisiaj są opłakiwane przez bliskich. Ale Ty wiesz swoje. Te bogate ku*wy nie będą Ci rozkazywać. Po miesiącach tortur zamiast Cię zabić zesłano Cię do pracy. Marne jedzenie i codzienne nowe narzędzia do sprawiania bólu stały się Twoją codziennością. Po tylu latach chciałbyś aby się to skończyło. Teraz wiesz, że mógłbyś to zrobić inaczej. Kiedy powiedziano, że wysyłają Cię za osadę w poszukiwaniu wody wiedziałeś, że to Twoja szansa na wyrwanie się z tego kurwidołka. Obiecano Ci oczyszczenie ze zbrodni jeśli wrocisz a wiesz, że sam tam możesz nie przetrwać. Pozatym trzeba dokończyć robotę...
    \end{tcolorbox}

    \newpage
    \bigskip

    Zaimponowanie zarządcy:

    \begin{tcolorbox}
        \ttfamily
        To Ciebie wybrał on, kochany i dobry przywódca. Nie możesz go zawieść! Zawsze byłeś na jego rozkazy. Od kiedy pamiętasz on uczył Cię chodzić i mówić. Czytać i pisać. Spełniałeś jego każde zadanie skrupulatnie i dokładnie. Czasami w zamian pozwalał Ci spać w łóżku a nie na ziemi. Kiedy się go słuchałeś życie dla Ciebie było dużo lepsze niż dla innych. Ale kiedy się nie słuchałeś wiedziałeś jak surowe są jego kary. Sam zgłosiłeś się aby pójść z tymi degeneratami i pokazać ile jesteś warty. Kiedy wyruszały pierwsze ekipy zabronił Ci iść. Mówił, że to nie dla Ciebie. Teraz powierzył Ci ważne zadanie. To Ty masz znaleźć wodę dla ludu i zaistnieć jako jego prawa ręka. Taka szansa nie zdarza się zawsze. nigdy o tym nawet nie marzyłeś. Taka szansa od zarządcy dla Ciebie. Jesteś wybrany, najlepszy!
    \end{tcolorbox}

    Badania i nauka:

    \begin{tcolorbox}
        \ttfamily
        "Ucz się synu ucz bo nauka to potęgi klucz." mawiał Twój dziad do Twojego ojca i ojciec do Ciebie. Z tego powodu uczyłeś się. Poznałeś matematykę, umiałeś składać w całość elementy o których inni mówili że to szmelc. Obsługę komputera w bibliotekę poznałeś w dwa lata. Po kolejnych trzech dostałeś się do rekordów dziennika pozostowianego przez Twojego dziadka. Myślałeś, że dzięki temu poprowadzisz do rozwoju osady, pomożesz zbuntować się przeciwko zarządcy. Zebrałeś grupę ochotników. Organizowałeś akcje antyrządowe. Umiałeś pomóc wszystkim gdyż poznałeś podstawy medycyny. Mimo walki z reżimem Ty dałeś anonimową wiadomość o skażeniu wody. Ale ta wiedza Cię zgubiła, bo ktoś był jednak mądrzejszy. Zamknięty w celi miałeś pracować dla zarządcy nad rozszyfrowaniem schematów, ale wiedza, którą tam ukryto w tym miejscu jest nie do odzyskania. Wiedział i o tym ten który wrzucił Cię do celi. Dlatego kazał Tobie iść i znaleźć żródło wody jeżeli nie ma tu jak jej oczyszczać. A jedynie Ty będziesz wiedział gdzie będzie czysta.
    \end{tcolorbox}

    \bigskip

    Po poznaniu swoich postaci i motywów, brzaskiem (słońce to sztuczna kula światła jeżdząca po suficie i odpowiednio zmieniająca kolor.) dnia następnego zostają pod obstawą zaprowadzeni na obżerza osady. Wskazany jest im kierunek do bramy i pod naporem siły są dalej wypychani.

    \section{W nieznane}

    Przedzierając sie przez, tutaj dosyś rzadkie, krzaki i drzewa przypominające te w lasach deszczowych po jakimś czasie gracze dotrą do płotu. Na płocie zawieszone są stare metalowe tabliczki z symbolem pioruna (mogą być zardzewiałe i prawie nieczytelne) oraz napisem "NIEBEZPIECZEŃSTWO". (Tu mogą odpaść pierwsi NPC, którzy dotknął płotu). Idąc wzdłuż płotu natrafią w końcu na bramę (Lub jeżeli ich postacie będą wiedzieć jak ominąć elektryczny płot to "skok przez płot"). Istnieje też szansa, że ustalą co stało się z kilkoma osobami z poprzedniej ekipy. Przy odpowiednim szczęściu uda im się znależć dziurę w płocie gdzie będą mogli przejść. Jeżeli jednak dotrą do bramy ich zadaniem będzie ją ominąć. Brama jest odizolowana od płotu i da się ją przeskoczyć. Zamek jest elektryczny. Siatka t o gruba siatka której nie da się poprostu przeciąć a sama brama osadzona na 10 metrowych stalowych balach i gruba na 1,5m stanowić będzie nie lada przeszkodę. Tutaj do dyspozycji graczy może być lina (i okazja do pozbycia się kolejnych NPCów). Za bramą jest już teren nieznany. Nikt nigdy z tamtąd nie wrócił z żadnymi informacjami. Wiadomo jednak, że dalej coś istnieje z powodu zapisków w dzienniku o stalowych maszynach, które z za płotu kiedyś przywoziły materiały.

    \normalfont
    Po godzinie dotarliście do wysokiego na 10m płotu. Płot ten to kilka metalowych belek postawionych pionowo i połączonych grubymi metalowymi kablami grubymi jak noga. (\emph{Coś na styl płotu z Jurrasic Park}). Na płocie zawieszone są podrdzewiałe tabliczki ze znaczkiem zygzaka pomalowanego na wyblakły żółty z zatartym napisem niebezpieczeństwo. 

    \begin{figure}
        \centering
        \def\svgwidth{\columnwidth}
        \input{fence.pdf_tex}
        \caption{Koncepcja płotu}
    \end{figure}

    - Idę przez niego - odzywa się jeden z waszych kompanów. \emph{Jeżeli chwyci za płot od razu zmieni się w czarny proszek na ziemi.} Wiedząc, że dotknięcie płotu grozi natychmiastową śmiercią szukacie sposobu na obejście go i rozumiecie teraz co działo się ze wszystkimi którzy go dotykali.



    Mineło kilka godzin od kiedy maszerując wzdłuż stalowej konstrukcji, kiedy w oddali zauważacie potężną metalową bramę. Zbliżając się bez problemu możecie określić, że brama, wysoka jak płot i gruma na 1,5 metra nie będzie łatwym celem. O dziwio nie ma na niej znaków ostrzegawczych. Jeden z was bierze i rzuca kawałkiem mięsa w nią. I nic się nie dzieje. Powtarza to samo z płotem obok i mięso wyparowuje. Bez zawachania napiera na bramę lecz ta ani drgnie. Przy bliższych oględzinach widzicie głębokie rysy na bramie kilka lekkich wgnieceń. Widać, że ktoś próbował ją sforsować.

    \bigskip
    \slshape
    Najprostrzym sposobem na ominięcie bramy będzie po prostu zarzucenie liny i przejście przez nią. (Kila osób może spaść z wysokości i zginąć.) Nie będzie to wymagające ani siły ani pomyślności.

    \bigskip
    \normalfont
    \emph{\textbf{Przekroczenie bramy.}}

    Za bramą flora gęstnieje. Bez odpowiednich narzędzi do tworzenia sobie drogi będziecie zmuszeni na długie i mozolne przeciskanie się między krzakami w nieznanym kierunku. Rośliny, które tu rosną są powykrzywiane z dziwnymi końcówkami. Nie widzicie jeszcze żadnej fauny, ale już po tym co zobaczyliście, wiecie, że to całkiem inny świat. Kiedy ostatnia osoba przeszła przez bramę zauważyliście, że na na jednym z słupków zaczyna świećić czerwone migające światło. Z nad bramy słychać charczące i niezrozumiele: Uciekinierzy zostaną poddani eksterminacji. \emph{(Jeżeli graczom nie uda się zrozumieć wypowiedzi jeden z ocalałych NPC powinien go zrozumieć i przetłumaczyć.)} Zaraz po tym w gaszczu w pobliżu płotu słychać poruszającą się głośną maszynę. \emph{(Czas na reakcję graczy. Ucieczka i rozproszenie lub walka z elektronicznym strażnikiem. Jeżeli jednak gracze przekroczą w jakiś sposób płot a nie bramę alarm się nie podniesie. Niezależnie od decyzji po walce/ucieczce: )} Wiecie, że nie będzie bezpieczie. Wybieracie kierunek marszu i ruszacie dalej. \emph{(Gracze powinni określić kierunek w którym chcą iść.)}

    \bigskip
    \normalfont
    \emph{\textbf{Przejście prze płot.}}

    Udało się wam ominąć niebezpieczny płot i jesteście teraz tam gdzie żaden z was wcześniej nie był. Flora za płotem gęstnieje. Za bramą flora gęstnieje. Bez odpowiednich narzędzi do tworzenia sobie drogi będziecie zmuszeni na długie i mozolne przeciskanie się między krzakami w nieznanym kierunku. Rośliny, które tu rosną są powykrzywiane z dziwnymi końcówkami. Słyszyćie jaką faunę ale jej nie widzicie. Nie martwiąc się o to co może was spotkać, choć wiecie, że nie może być bezpieczie wybieracie kierunek dalszej podróży. \emph{(Gracze powinni określić kierunek w którym chcą iść.)}

    \section{Północ, zachód i południe}

    Idąc w tym kierunku po godzinie przedzierania się przez krzaki i wystraszeniu jakichś zwierząt lub mutantów powoli Sbliżacie się do płaskiego wielkiego "malowidła". Kiedy podchodzicie do niego staje się ono jakby jaśniejsze i się porusza. Nie macie pojęcia co to jest. W dotyku jest ciepła lecz nie potraficie powiedzieć gdzie się kończy a gdzie zaczyna. W tym momencie w głowie przełącza się wam pstryczek. Może to kolejny rodzaj płotu? \emph{(Możesz psytać graczy czy chcą go próbować ominąć.)} Po kilku godzinach prób zaczepienia gdzieś liny i chodzenia wzdłóż ściany zmęczeni całym dniem postanawiacie odpocząć. Kiedy już zjedliście i napoiliście się i ruszacie w dalszą podróż.

    \section{Południowy wschód}

    Przebijając się przez gęste zarośla 

    \section{Wschód}





    \end{document}
