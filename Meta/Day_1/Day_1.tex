\documentclass{article}

    \usepackage{a4wide}
    \usepackage{polski}
    \usepackage[utf8]{inputenc}
    \usepackage{float}
    % \usepackage[greek]{babel}
    % \usepackage[LGR]{fontenc}

    \newcommand{\paragraphx}[1]{
        \paragraph{\Large{{#1}}}\mbox{}

    }

    \title{\Huge{Day 1}}
    \author{Arkadiusz Popczak}
    \begin{document}
    \maketitle

    \paragraphx{Przedmowa}
    Rok 1231 kalendarza Rodgera I: W wiosce przybywa coraz więcej ludzi. Dziwne stalowe istoty przestały dostarczać nam jedzenia i wody pod żelazną bramą. Dostajemy w nich narzędzia oraz rzeczy z księgami, które rozumieją tylko starsi. Uczą nas jak z nich korzystać. Jeżeli nie będziemy potrzebować stalowych istot aby przetrwać może będziemy mogli utrzymać przyrozt populacji.

    Rok 1239 kalendarza Rodgera I: Nasze uprawy i hodowle zwierząt przynoszą efekty. Populacja nasza zwiększyła się dwukrotnie choć wiele dzieci nadal ginie przy porodzie. Starszyzna dostała dziwne rysunki od stalowych. Podobno będziemy teraz budować domy z kamienia. \\
    Rok 1265 kalendarza Rodgera I: Starszyzna została zabita podczas buntu. Piszę więc swoje ostatnie słowa aby przekazać informacje o urządzeniach które zostały do tego użyte. Schematy oraz egzemplarze jakie dostaliśmy zostały zniszczone. Jeżeli ta broń zagłady zostanie rozpowszechniona przyszłość miasta jest dla mnie zagadką.

    Rok 1267 kalendarza Rodgera I: 435 dzień buntu. Jesteśmy prawie pewni, że tamci zdobyli schematy broni. Wczoraj Andre zobaczył otwartą bramę i stalowych przez nią wchodzących. W nocy słyszeliśmy krzyki po drugiej stronie barykady. Od tamtej pory ataki ustały. Przygotowujemy się na najgorsze choć niektórzy twierdzą, że to my wygraliśmy.

    Rok 0 kalendarza po Odkryciu: Odnależliśmy dzisiaj ten dziennik i postanowiłem go kontynuować. Nie mam pojęcia co to za broń ani za stalowe istoty, ale jestem pewien, że zabiły naszych przodków. Nikt nie wrócł żywy spod bramy o której mowa. Jeżeli to ta sama brama którą my napotkaliśmy. Ann oczekuje kolejnego dziecka. Liczba osób: 121.

    Rok 30 kalendarza po Odkryciu: Zbliżają się moje ostatnie dni. Osada przeistacza się w coś większego. Mamy dostęp do wody oraz jedzenia. Wszystkie schematy które znalazłem ukryłem aby nie dostały się w niepowołane ręce. Jeżeli mielibyśmy być zmasakrowani tak jak poprzednicy. Jeżeli czytasz ten dziennik wiedz, że nie przynoszą one nic dobrego. Populacja: 430.

    Rok 56 kalendarza po Odkryciu: Znalazłem schematy ale nie umiem ich odczytać. dziwne rysunki pomogły nauczyć mi się jak działają niektóre ze znalezisk. Zachowam je na wypadek jakiegoś buntu czy ataku. Pora przejąć władzę nad miastem.
    
    Rok 143 kalendarza po Odkryciu: Zaczyna brakować wody. Staw który zawsze był pełen wysycha. Reszta wody jest zanieczyszczona. Technicy, którzy analizują sytuację nie znają nadal ku temu powodu. Pierwsze wyprawy za stalową barierę nie wracają już od kilku miesięcy. Musimy wysłać kolejną. Jeżeli nie znajdziemy nowego źródła wody czeka nas zagłada. Populacja 1540.

    \paragraphx{Problemy z wodą}
    - Wasza 16 została wybrana aby wyjść z naszego 
    \end{document}
